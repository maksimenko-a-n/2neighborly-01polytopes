\documentclass[12pt]{article}
\usepackage{cmap}
\usepackage[T2A]{fontenc}
\usepackage[utf8]{inputenc}
\usepackage[russian,english]{babel}

\usepackage{amsmath, amsfonts}
\usepackage{mathtools}
\DeclareMathOperator{\conv}{conv} %Convex hull
\DeclareMathOperator{\cone}{cone} %Conical hull
\DeclareMathOperator{\ext}{ext} %The set of extreme points

\sloppy

\begin{document}

\title{The database of 2-neighborly 0/1-polytopes\\ of dimensions 6 and 7}
\author{Aleksandr N. Maksimenko}
\maketitle

\paragraph{2-neighborly 0/1-polytopes.}
We consider only convex polytopes~\cite{Grunbaum:2003}.
A~convex polytope of dimension~$d$ is called a \emph{$d$-polytope}.
A face of $P$ is called \emph{$k$-face} if its proper dimension is equal to $k$.
A 0-face of a $d$-polytope is called a \emph{vertex}, a 1-face is called an \emph{edge}, a $(d-1)$-face is called a \emph{facet}. The face lattice of a polytope is the set of all its faces ordered by inclusion.
Two polytopes are \emph{combinatorially equivalent} (has the same \emph{combinatorial type}) if they have isomorphic face lattices.
It is well known that the combinatorial type of a~polytope $P$ with vertices $\{v_1, \dots, v_n\}$  and facets $\{F_1, \dots, F_k\}$ is uniquely determined by its facet-vertex incidence matrix $M = (m_{ij}) \in \{0,1\}^{k \times n}$, where $m_{ij} = 1$ if facet $F_i$ contains vertex $v_j$, and $m_{ij} = 0$ otherwise.
Thus, polytopes are combinatorially equivalent iff their facet-vertex incidence matrices differ only by column and row permutations.

Let $f_i$, $i=0,1,\dots,d-1$, be the number of $i$-faces of $P$.
The vector $(f_0, f_1, \dots, f_{d-1})$ is \emph{f-vector} of $P$.
A polytope $P$ is \emph{2-neighborly} if any two vertices form an edge of $P$,
i.e. $f_0 (f_0 - 1) = 2 f_1$.
A convex polytope $P$ is called a \emph{0/1-polytope} if its set of vertices $X = \ext(P)$ is a subset of $\{0,1\}^d$. All 0/1-polytopes can be splitted into \emph{0/1-equivalence classes} (\emph{0/1-classes})~\cite{Ziegler:2000}.
If two polytopes are 0/1-equivalent then they are combinatorially equivalent.

In the database, there are listed all 2-neighborly 0/1-polytopes of dimensions 6 and 7.
For $d=7$, we cann't provide the entire database explicitly, since it occupies about 1TB.
Instead of this, we provide only 7-polytopes with the minimal and maximal number of facets (for every fixed number of vertices). f-vectors of these polytopes are listed in the file \texttt{7dminmax.fv}.
However, all f-vectors of 2-neighborly 0/1-polytopes of dimension 7 are listed in the file \texttt{7d.fv}.
If you want to work with the entire database, don't hesitate to send a request to \texttt{maximenko.a.n@gmail.com}

\paragraph{The database structure.}
The folder \texttt{6d} has 11 subfolders with names \texttt{03v}, \dots, \texttt{13v}.
The number in subfolder's name is the number of vertices of a polytope.
Every subfolder contains subsubfolders with names \texttt{Nf}, where \texttt{N} is the number of facets of a polytope.
Thus, \texttt{6d\textbackslash11v\textbackslash023f} contains descriptions of all polytopes with 11 vertices and 23 facets.

All polytopes are splitted by f-vectors.
Every set of polytopes with the same f-vector occupies two files.
For example, the file\\
 \texttt{6d\textbackslash11v\textbackslash023f\textbackslash11v023f-0136-0167-0098}\\
contains information of combinatorial types of 2-neighborly 0/1-polytopes with f-vector $(11, 55, 136, 167, 98, 23)$.
The file \\
\texttt{6d\textbackslash11v\textbackslash023f\textbackslash11v023f-0136-0167-0098.01}\\
contains all 0/1-classes of such polytopes.

The list of all f-vectors of 6-polytopes is in the file \texttt{6d.fv}.
Every line in the file is a short description of the set of polytopes with one f-vector.
For~example:\\
\centerline{\texttt{c4p1( 9 36 80 102 70 21)C8P15594}}
The parameter \texttt{c4} means that the set contains a 4-simplicial polytope.
(The set can also contains polytopes that are not 4-simplicial.)
The parameter \texttt{p1} means that it contains a pyramid.
The \texttt{p2} means that one of the facets (called \emph{base}) of a polytope is adjacent with all the other facets, and 2 vertices do not belong to the facet (base).
In parentheses, there is the f-vector.
After \texttt{C}, there is the number of combinatorial types.
After \texttt{P}, there is the number of 0/1-classes.

\paragraph{How to work with the database.}
The description of polytopes and their facet-vertex incidence matrices can be retrieved by the program \texttt{read.c}. The program works in command line mode. For example, the command\\
\centerline{\texttt{read 6d.fv -v 12 -all}}
will write into the file \texttt{6d.fv-12v.txt} the list of all incidence matrices and all 0/1-classes of 2-neighborly 0/1-polytopes of dimension~6 with 12~vertices.

The first parameter must be the name of the file with f-vectors. (You can manually remove unwanted lines (f-vectors) from the file.)
You can also use some additional parameters:

\texttt{-v} M --- select only polytopes with M vertices.

\texttt{-f} N --- select only polytopes with N facets.

\texttt{-fv} ---    write only f-vectors (lines from the input file).

\texttt{-all} ---   write all representatives for a combinatorial type (only one be written by default).

\texttt{-notinc} ---   don't write facet-vertex incidence matrix.

By default, for every f-vector there will be listed all combinatorial types. One combinatorial type will be represented by a facet-vertex incidence matrix and one 0/1-polytope (the list of vertices).

\paragraph{The format of files.}
For every f-vector there are two files: the file with 0/1-classes has extension \texttt{.01}, the file with combinatorial types has no extension. For~example,\\
\texttt{6d\textbackslash11v\textbackslash023f\textbackslash11v023f-0136-0167-0098.01}\\
and\\
\texttt{6d\textbackslash11v\textbackslash023f\textbackslash11v023f-0136-0167-0098}

The first file has $N_p$ records, where $N_p$ is the number of 0/1-classes with the given f-vector.
Every record is a 0/1-polytope wich is a representative of the appropriate 0/1-class.
A 0/1-polytope with $N_v$ vertices is written as $N_v$ bytes,
since every vertex is a 0/1-vector with $d$ coordinates, $d \le 7$.
All 0/1-classes in the file ordered by the combinatorial type.

The second file has $N_c$ records, where $N_c$ is the number of combinatorial types with the given f-vector.
One record has the following structure:\\
\centerline{\texttt{[incidence matrix][N1][N2][polytope]}}
The facet-vertex incidence matrix occupies $N_f$ blocks, where $N_f$ is the number of facets.
One block is one row of the matrix. If the number of vertices $N_v \le 16$ then one block occupies 2 bytes, otherwise~--- 4 bytes.
The polytope occupies $N_v$ bytes (the same as in \texttt{*.01}).
\texttt{N1} and \texttt{N2} are the positions of the first and the last polytopes in the file \texttt{*.01} with this combinatorial type. 


\begin{thebibliography}{99}
 
\bibitem{Aichholzer:2000} 
  O. Aichholzer.  
  Extremal properties of 0/1-polytopes of dimension~5.
  In Polytopes -- Combinatorics and Computation. Birkhauser, 2000, pp.~111--130.

%\bibitem{Gillmann:2006} 
%	R. Gillmann. 
%	0/1-Polytopes: Typical and Extremal Properties. 
%	PhD Thesis, TU Berlin, 2006.
    
\bibitem{Grunbaum:2003} 
  B. Gr\"unbaum.
  Convex polytopes. Second edition.
  Springer, 2003. 
		
\bibitem{Ziegler:2000} 
  G.M. Ziegler.
  Lectures on 0/1-Polytopes.
	In Polytopes --- Combinatorics and Computation. Birkhauser, 2000, pp.~1--41.

\end{thebibliography}

\end{document}

